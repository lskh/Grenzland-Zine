
Zufallstabellen und Generatoren faszinieren mich sehr. Ich blättere
manchmal durch \textit{Yoon-Suin} oder Matt Finch's \textit{Tome of
Adventure Design} und staune über die vielen \textit{potentiellen}
Abenteuer, die da im Aether herumwabern ... unendlich viele. 
Es ist dieses \textit{Potential}, was
mich anspricht. Ein Blick in das Multiversum, der sich mit Hilfe
einiger unscheinbarer magischer Würfel öffnet. Seltsamerweise macht
es für mich auch einen großen Unterschied, ob ich aus einer Tabelle
willkürlich auswähle, oder die Würfel entscheiden lasse. Auszuwählen
fühlt sich irgendwie nach Schummeln an, und die Ergebnisse werden am
Ende leicht klischeehaft. Wenn ich würfle, fühle ich
mich den Würfeln verpflichtet, und nicht selten wollen diese
launischen Biester Dinge aus den Tabellen auswählen, die 
auf den ersten Blick überhaupt nicht zusammen zu passen scheinen. So
würde man das niemals willkürlich auswählen. Und genau da liegt das
Gold. Wenn mich die Würfel \textit{zwingen} zu erklären, warum in
meiner pseudomittelalterlichen Fantasy-Welt eiförmige Pods durch den
Himmel gleiten, und warum diese armdicken Kriechtiere, die aussehen wie
kuschelig bepelzte Raupen in Wirklichkeit intelligente Pflanzen
sind, und wenn ich mich frage, mit welchen perfiden Tricks der
Biochemie diese dann auch noch 3W6 Schaden verursachen, dann beginnt
das kreative Worldbuilding ... der Blick in den Astralraum.

Wie gesagt, das Material war eigentlich schon da. Also ziehe ich
mich nach diesem Vorwort aufs editieren zurück: Der 
\textit{Deskriptive Dungeon-Generator} und der
\textit{Kampagnen-Generator} sind Projekte aus dem Manuskript für
mein \textit{Menschen \& Magie}-Spielleiterbuch. Letzteres ist
einerseits als deutschsprachiges Drop-In 
für Band 3 der Original Fantasy-Rollenspielregeln von 1974 gedacht,
andererseits als ``Bastelbuch'' für Do-it-yourself-Abenteuerspiele im
Allgemeinen. Man kann damit also den Klassiker spielen, oder 
sich Inspiration holen, um ein eigenes maßgefertigtes Spiel im
Stil der \textit{Old School Renaissance} zu entwickeln. Leider 
ist es noch lange nicht fertig, im Moment vielleicht so bei 60 \%, was
auch daran liegt, dass ich parallel am zugehörigen Monsterbuch
arbeite. Naja, und der \textit{day job} ... aber
die Generatoren funktionieren ja schon, und sind auch prima ohne das
Spielleiterbuch zu gebrauchen. Die sind ohnehin alle
system-agnostisch, können also mit jedem beliebigen
Rollanspielsystem benutzt werden. Die \textit{46.656 Psychedelische
Landschaften} schließlich sind die deutsche Übersetzung eines
Blog-Artikels, den ich vor einiger Zeit als Fingerübung und
Vorbereitung für eine Traveller-Kampagne geschrieben habe.  

Aber wo wir eben beim Thema ``system-agnostisch'' waren, 
da bin ich sowieso ein Fan von. Leute, schreibt mehr
system-agnostisches Zeugs! Ich finde damit bleibt alles viel
flexibler. Und gute Regelwerke sollten eine Anleitung enthalten, wie
man z.B. aus einer bloßen Monsterbeschreibung, vielleicht sogar nur
aus einem coolen Bild von einem Monster, ein mit Spielwerten
ausgestattetes, funktionierendes, regelkonformes Spielelement erzeugt.
\textit{Dungeon World} hat das z.B. hervorragend gelöst. Aber auch
mit den alten \textit{Classic Traveller}-Regeln funktioniert das
sehr gut. Und ansonsten habe ich bei \textit{Ken \& Robin} den Tipp
gehört, dass man im Zweifel immer die die \textit{stats} von einem
Braunbären nehmen kann. Den Bär kann man dann beliebig umlackieren,
und die sonstigen Details des
Monsters frei dazu erfinden.


Was kommt in der nächsten Ausgabe des \textbf{Grenzland}? Ich denke
ein paar system-agnostische Kurzabenteuer. Das Material ist schon
da, es muss nur editiert werden, aber wer weiß, wann ich dazu komme
(hier zwinkernden Emoji denken). Erstmal viel Spaß mit den
Kontent-Generatoren!


\section{Ein deskriptiver Dungeon-Generator}
\by{Wanderer Bill}

Die meisten Dungeon-Generatoren setzten voraus, dass man auf
Tabellen würfelt, und dann anfängt zu zeichnen. Und genau deswegen
funktionieren sie meiner Meinung nicht gut, weil man sich sofort in
Details verstrickt. Das kann ganz nett sein, und die Kreativität
anspornen, aber zum improvisierten Spiel ist das nicht geeignet. Die
Spieler sollen einem ja nicht beim Game-Prep zugucken.

Wenn ich einen Dungeon komplett improvisiere, kann ich einfach
erzählen: ``nachdem Ihr die steile rutschige Treppe hinter Euch
gelassen habt,  betretet ihr einen kleinen Raum aus grauem,
feuchtem Mauerwerk, vielleicht 4 x 6 Meter groß. In der Nordwand und
der Südwand seht ihr im flackernden Licht eurer Laternen jeweils
eine verschlossene Tür.  Hinter der Tür in der Südwand ist ein
kratzendes Geräusch nicht zu überhören. Was tut ihr?'' - Bis die
Spieler 
ausdiskutiert haben, wer jetzt was macht, habe ich schon einen Blick
auf eine Tabelle mit Zufallsbegegnungen geworfen, und \textit{ad
hoc} entschieden, dass hinter der Tür in der Südwand ein alter Lagerraum ist,
und hinter der in der Nordwand eine weitere Treppe in einen
stickigen dunklen Gang führt, der sich dann weiter aufzweigt. Dann
warte ich ab, was die Spieler vorhaben, und lasse eventuell die ein
oder andere Probe würfeln. Von
mir aus können sich die Spieler nach meiner Beschreibung eine Karte
zeichnen. Mir reicht ein Diagramm mit Stichworten, Symbolen und
Pfeilen.

Aufbauend auf dieser theoretischen Fingerübung, habe ich folgenden
Dungeon-Generator entwickelt. Es ist ein Dungeon-Generator zum
Lesen und Vorlesen. Der Titel jeder Liste weißt darauf hin, mit
welchem Würfel jeweils gewürfelt werden muss. Mit Größen- und
Längenangaben darf man es nicht zu genau nehmen, es sind eben
narrative Maße, und nicht selten täuscht die Wahrnehmung die
tapferen Abenteurer ja auch.

\subsubsection{I - Der Eingang \ldots{} (1W6)}

\begin{tabularx}{\columnwidth}{cZ}
1 &  \ldots{} ist eine unscheinbare natürliche Höhle, die verdeckt von
  dichtem Gestrüpp kaum zu erkennen ist.\\
2 & \ldots{} ist ein uraltes gemauertes Portal aus grob gehauenen, grauen
  Steinquadern. Vom Eckstein herab schaut Euch eine hämisch grinsende
  dämonische Fratze an.\\
3 &  \ldots{} ist durch ein geschmiedetes Gittertor verschlossen. Es ist
  rostig und mit spitzen Zacken versehen. An einer der Eisenzacken hängt
  ein Stofffetzen, der sich im Wind träge etwas hin und her
  bewegt.\\
4 &  \ldots{} sieht aus wie ein stillgelegter Bergwerksstollen. Der Weg in
  den Stollen ist mit ein paar Holzbohlen blockiert und auf einem Schild
  steht in verblichener Farbe ``Lebensgefahr''.\\
5 &  \ldots{} ist eine Spalte im Stamm einer gewaltigen tausendjährigen
  Eiche. Mühsam müsst Ihr Euch mit Euren Habseeligkeiten durch den Spalt
  quetschen, bevor Ihr auf einer engen Wendeltreppe etwa 6 Meter weit in
  die Tiefe hinab klettern könnt.\\
6 &  \ldots{} wird sichtbar, als Ihr die große Steinplatte mit Mühe zur
  Seite schieben könnt. Im flackernden Licht Eurer Fackeln sehr Ihr wie
  der jahrhunderte alte Staub auf die steinernen Treppenstufen rieselt,
  die vor Euch in die Dunkelheit führen.\\
\end{tabularx}

\subsubsection{II - Wenige Meter hinter dem Eingang \ldots{} (1W8)}

\begin{tabularx}{\columnwidth}{cZ}
1  & \ldots{} ändert sich die Struktur des Bodens und der Wände. Der Gang
  vor Euch ist aus grauen Steinquadern gemauert, etwa 3 Meter breit und
  2 Meter hoch. Ihr könnt in Zweierreihen gehen, wenn ihr wollt. Alle
  paar Meter seht ihr alte schmiedeeiserne Fackelhalter an den Wänden.
  Hier und da seht ihr auch noch verkohlte Stümpfe von Fackeln. Ihr hört
  Eure Schritte deutlich hallen, aber wenn ihr kurz inne haltet könnt
  ihr hinter Euch noch ganz leise den Wind wehen hören.
\\
2 & \ldots{} kommt ihr in einen etwa drei Meter breiten, grob aus dem
  Erdreich gegrabenen Gang. Hier und da sind Kratzspuren an den Wänden
  zu sehen. Welche Kreaturen diesen Gang wohl mit ihren Krallenhänden
  gegraben haben mögen? Im flackernden Licht Eurer Fackeln seht ihr ab
  und zu kleine Spinnen und Tausendfüßler im rissigen Lehm verschwinden.
  Plötzlich hört Ihr vor Euch ein Geräusch [Probe auf
  Zufallsbegegnung].
\\
3 & \ldots{} geht der Gang in eine natürliche Höhle über. Das Vorankommen
  ist mühsam, denn es geht über den felsigen Boden ständig auf und ab.
  Hier und da seht ihr Gruppen von Tropfsteinen, und ab und zu hört Ihr
  das Geflatter von Fledermäusen. Plötzlich rutscht [ein zufällig
  bestimmter Charakter - z.B. der mit dem niedigsten
  Geschicklichkeitswert] aus, und rutscht mit lautem Gepolter einen
  schrägen Fels hinunter. [Ein Rettungswurf entscheidet, ob er sich dabei
  verletzt.]
\\
4 & \ldots{} kommt ihr zu einer etwa drei Meter breiten steinernen Treppe,
  die weiter nach unten ins Dunkel führt. Als Ihr die Treppe
  hinabsteigt, habt Ihr das Gefühl, dass Eure Schritte noch lauter
  hallen als zuvor. Die Luft ist stickig. Als Ihr am unteren
  Treppenabsatz ankommt, hört ihr weit vor Euch das Knarren von rostigen
  Türangeln und dann einen Knall der nur langsam verhallt.
\\
6 & \ldots{} bemerkt ihr plötzlich einen Windstoß. Eure Fackeln beginnen
  heftig zu flackern und drohen zu erlöschen. Bei einem Wurf von 3 oder
  weniger auf 1W6 steht ihr einen Moment später im Dunklen. Wenn Ihr
  Euch konzentriert, könnt ihr irgendwo vor Euch, in einiger Distanz,
  vorsichtige Schritte hören. [Probe auf Zufallsbegegnung]\\
7 & \ldots{} erreicht Ihr einen in unheimliches Licht getauchten Gang, der
  quer zu Eurer Marschrichtung von links nach rechts verläuft. Der Gang
  ist etwa 3 Meter breit, und aus grauen, grob behauenen Steinen
  gemauert. Etwa alle 5 bis 6 Meter ist der Gang mit einer Fackel
  erleuchtet, die in einem schmiedeeisernen Fackelhalter steckt.
  \\
\end{tabularx}
\begin{tabularx}{\columnwidth}{cZ}
8 & \ldots{} öffnet sich vor Euch ein runder Raum, in dessen Mitte
eine steinerne Wendeltreppe in die Tiefe führt. Von unten kommt Euch
ein moderiger, kühler Windhauch entgegen. Nachdem Ihr der Treppe
etwa 6 Meter weit nach unten gefolgt seit, erreicht Ihr einen
Treppenabsatz vom dem ein Gang abzweigt. Die Treppe führt von hier
aus weitere [1W20] Etagen nach unten.\\

\end{tabularx}

\subsubsection{III - Nach 10 Minuten vorsichtiger Erkundung \ldots{} (1W20)}


\begin{tabularx}{\columnwidth}{cZ}
\def\labelenumi{\arabic{enumi}.}
1 & 
  \ldots{} erreicht Ihr eine verschlossene Tür. Sie ist aus schweren,
  etwas feuchten Holzbohlen zusammen gesetzt und mit schmiedeeisernen
  Bändern verstärkt. Sie kann nur mit einer entsprechenden erfolgreichen
  Probe geöffnet werden.

  Bei 1 auf W6 könnte hinter der Tür eine Zufallsbegegnung statt finden.
  Durch eine erfolgreiche Probe auf Lauschen kann Überraschung vermieden
  werden.
\\
2 & 
  \ldots{} zweigt nach rechts ein Gang ab.

  Bei 1 auf W6 kommt es zu einer Zufallsbegegnung.

  Bei einer entsprechenden erfolgreichen Probe ist aus Richtung der
  Abzweigung ein Geräusch zu hören, auch wenn es nicht zu einer
  Begegnung kommt.
\\
3 & 
  \ldots{} zweigt nach links ein Gang ab.

  Bei 1 auf W6 kommt es zu einer Zufallsbegegnung.

  Bei einer entsprechenden erfolgreichen Probe ist aus Richtung der
  Abzweigung ein Geräusch zu hören, auch wenn es nicht zu einer
  Begegnung kommt.
\\
4 & 
  \ldots{} öffnet sich der Gang durch einen runden Torbogen in eine
  achteckige Kammer etwa 9 x 9 Meter groß. In der gegenüberliegenden
  Wand, und an den beiden geraden Wänden links und rechts, seht ihr
  ebenfalls Torbögen, die mit Fresken wiederlicher Teufelsfratzen
  verziert sind. Auf den Boden der Kammer ist mit brauner Farbe ein
  thaumaturgischer Kreis gezeichnet. Vor den vier diagonalen Wänden
  hängen an jeweils drei Messingketten Weihrauchfässer von der Decke
  herab. Die Luft ist kalt, und der Raum scheint länger nicht mehr
  benutzt worden zu sein. Dennoch liegt ein Spur von Weihrauch in der
  Luft.

  Bei 5 oder 6 auf 1W6 steht vor einer der diagnonalen Wände eine
  schwere Truhe. Sie ist mit Sicherheit durch irgend einen
  Fallenmechanismus gesichert, im Zweifel durch Giftpfeile. Sind die
  Spieler nicht vorsichtig, entscheidet ein Rettungswurf, ob ein
  Charakter beim Versuch die Truhe zu öffnen tödlich verunglückt. In der
  Truhe finden sich 1 - 4 magische Elixiere, 1 Schriftrolle, ein
  magischer Dolch. Unter einem doppelten Boden schließlich auch noch
  30-180 Goldmünzen.
\\
5 & 
  \ldots{} springt die Wand auf der rechten Seite etwas zurück. In der
  etwa 3 Meter breiten, und 1 Meter tiefen Nische seht Ihr eine
  verschlossene Tür.

  Ein Wurf von 1 auf 1W6 entscheidet, ob hinter der Tür eine Begegnung
  lauert.
\\
6 & 
  \ldots{} springt die Wand auf der linken Seite etwas zurück. In der
  etwa 3 Meter breiten, und 1 Meter tiefen Nische seht Ihr eine
  verschlossene Tür.

  Ein Wurf von 1 auf 1W6 entscheidet, ob hinter der Tür eine Begegnung
  lauert.
\\
\end{tabularx}
\begin{tabularx}{\columnwidth}{cZ}
7 & 
  \ldots{} erreicht ihr erneut eine Treppe. WÜrfel 1W4: (1) Diese führt
  geradeaus in die Tiefe. Ihr bemerkt einen kühlen Luftzug, der Euch von
  unten entgegen weht. (2) Diese zweigt nach rechts vom Gang ab, und
  führt nach unten. (3) Diese zweigt nach links vom Gang ab und führt
  nach oben. (4) Diese führt nach oben. Ihr von oben hört Ihr etwas
  Rauschen. Vielleicht Wind in der Ferne.
\\
8 & 
  \ldots{} führt der Gang eine grobe natürliche Treppe über mehrere
  Meter nach Unten. Dann öffnet sich der Weg in eine große natürliche
  Höhle. Unmöglich bei dem schwachen Licht zu erkennen wie groß die
  Höhle ist. Ihr hört Fledermäuse und Eure Schritte hallen mit langem
  Echo wieder.
\\
9 & 
  \ldots{} erreicht Ihr eine breite doppelflügelige Tür, die mit
  seltsamen Runen beschriftet ist, und mit einer schweren eisernen Kette
  verriegelt ist.
\\
10 & 
  \ldots{} erreicht Ihr eine T-Kreuzung. Ihr könnt nach links oder nach
  rechts weiter gehen.

  Falls erfolgreich gelauscht wird, kann in einem der Gänge ein leises
  Geräusch gehört werden. WÜrfel auf der passenden
  Zufallebegegnungstabelle, worum es sich handeln könnte.

  Bei 1 auf 1W6 kommt es nach wenigen Metern zu einer entsprechenden
  Begegnung.
\\
11 & 
  \ldots{} bemerkt ihr im Gang vor Euch ein rostiges Fallgitter. Bei 1-3
  auf 1W6 ist es geschlossen, und kann nur mit einer schweren
  Stärkeprobe geöffnet werden. Bei 4-6 ist es geöffnet.

  Hinter dem Fallgitter öffnet sich der Gang nach wenigen Metern durch
  einen unheimlich schimmernden Torbogen in eine große längliche
  rechteckige Halle. Die Halle ist sicher 10 x 30 Meter groß, und die
  gewölbte Decke wird auf beiden Seiten durch Reihen von Säulen
  abgestützt. Am gegenüberliegenden Ende der Halle seht Ihr ein flaches
  Steinpodest, welches auf beiden Seiten mit Statuen gesäumt wird. Vor
  den Statuen stehen Kohlebecken.

  An beiden Längsseiten der Halle finden sich jeweils 1-4 verschlossene
  Türen.

  Hier kann man sich natürlich sehr gut auch einen Altar vorstellen, der
  auf dem Podest steht. Wenn es in die Handlung passt, findet hier
  vielleicht auch gerade ein unheiliges Ritual oder eine Opferung statt.
  Die gläubigen, die dem Ritual beiwohnen können jeder Art sein.
  Vielleicht sind es gar kein Menschen, sondern ungewöhnliche Humanoide,
  oder es ist gar nicht zu erkennen, um wen es sich handelt, da alle
  lange Kutten oder Mäntel mit tief ins Gesicht gezogenen Kapuzen
  tragen.
\\
\end{tabularx}
\begin{tabularx}{\columnwidth}{cZ}
12 & 
  \ldots{} bemerkt ihr einen unangenehmen Geruch. Irgendwie nach Tier,
  aber auch nach Verwesung. Vorsichtig den Gang entlang spähend, seht
  ihr links/rechts eine etwa vier Meter durchmessende Öffnung in der
  Wand, die in eine natürliche Höhle zu münden scheint.

  Hier haben die Charaktere das Lager eines Monsters erreicht. Ein Wurf
  auf der entsprechenden Zufallsbegegnungstabelle entscheidet, worum es
  sich handelt.
\\
13 & 
  \ldots{} erlischt Eure Lichtquelle. Ihr müsst sie nachfüllen, oder
  eine neue entzünden.
\\
14 & 
  \ldots{} hört Ihr von irgendwo vor Euch ein Poltern.
\\
15 & 
  \ldots{} hört Ihr hinter Euch ein kurzes Fauchen und dann einen
  dumpfen Knall. Danach ist es still. Ihr hört nur Euer Herzklopfen,
  welches Euch bis zum Hals hoch schlägt.
\\
16 & 
  \ldots{} habt Ihr plötzlich das Gefühl, dass der Gang leicht ansteigt.
\\
17 & 
  \ldots{} habt Ihr plötzlich das Gefühl, dass der Gang leicht abfällt.
\\
18 & 
  \ldots{} gehen die Wände plötzlich in spiegelndes, bläulich glänzendes
  Schillern über. Als Ihr Euch herum dreht, seht ihr auch nichts weiter
  als diese seltsam glänzenden Wände. Der Gang, durch den Ihr gekommen
  seid, ist nicht mehr zu sehen \ldots{}
\\
19 & 
  \ldots{} bemerkt Ihr im Schatten vor Euch eine Bewegung: Zeit für eine
  Zufallsbegegnung.
\\
20 & \ldots{} werdet Ihr plötzlich von einer Begegnung überrascht.
\end{tabularx}

\subsubsection{IV - Hinter der Tür \ldots{} (1W10)}

\begin{tabularx}{\columnwidth}{cZ}
1 & 
  \ldots{} kommt Ihr in einen Gang, der nach etwa 6 Meter nach links /
  rechts abzweigt. Würfel einen W20 und gehe zurück zu Liste 3.
\\
2 & 
  \ldots{} seht Ihr einen rechteckigen Raum, vielleicht 9 x 9 Meter
  groß. Es sind keine weiteren Türen zu sehen. Im schwachen Licht seht
  Ihr einen Tisch und ein paar Stühle in der Mitte des Raumes. Außerdem
  stehen an den Wänden 1-6 grob gezimmerte Bettgestelle. Die Luft riecht
  ranzig.

  In diesem Raum könnte ein kleiner Schatz zu finden sein (z.B.
  Schatztyp E).

  Und leider auch ein paar unangenehme Dungeonbewohner, wie zum Beispiel
  ein Aaskriecher, irgendeine Art von Blob oder Schleim, Ratten oder
  Tausendfüßler, die Krankheiten übertragen, oder eine große Giftspinne
  könnte diesen Raum zu ihrem Lager gemacht haben.

  Falls die Abenteurer nach einer Gruppe von Räubern oder sonstigen NSCs
  suchen, könnte dieser Raum deren Unterschlupf sein.
\\
3 & 
  \ldots{} findet Ihr eine Treppe, die nach unten führt. 1W20 und zurück
  zu Liste 3.
\\
4 & 
  \ldots{} seht Ihr einen niedrigen gewölbten Raum, in dem mehrere
  Truhen stehen. Leider hört Ihr noch bevor Ihr den Raum wirklich
  betreten könnt ein Klicken und ein scharfes Zischen, und nur ein
  erfolgreicher Rettungswurf kann verhindern, dass der erste Charakter,
  der durch die Tür tritt von mehreren Speeren getroffen wird, die 2W6
  Punkte Schaden verursachen.

  In den Truhen dieses Raumes könnte sich ein größerer Schatz, z.B. des
  Schatztyps E finden. Allerdings wird das Rumoren der Charaktere in
  diesem Raum in jedem Fall eine Zufallsbegegnung auslösen, die
  eintritt, sobald die Charaktere den Raum verlassen.

  Außerdem könnten einige der Truhen mit weiteren Fallen oder Flüchen
  gesichert sein, oder - ein Klassiker: bei einer der Truhen handelt es
  sich um eine Mimik, ein magisches Raubtier, welches die Gestalt von
  Gegenständen annehmen kann, und so auf seine Opfer lauert.

  Was immer hier angetroffen wird, ist in jedem Fall sehr unglücklich
  darüber, dass der Raum geplündert wird.
\\
5 & 
  \ldots{} geht ein gemauerter Gang für etwa 9 Meter geradeaus weiter,
  dann erreicht Ihr eine T-Kreuzung. 1W20 und zurück zu Liste 3.
\\
\end{tabularx}
\begin{tabularx}{\columnwidth}{cZ}
6 & 
  \ldots{} findet Ihr einen großen rechteckigen Raum, etwa 6 x 12 Meter
  groß, den Ihr von einer der Längsseiten betretet. In der Ecke
  links/rechts, schräg gegenüber von Euch sehr ihr eine verschlossene
  Tür. Die Wände des Raumes sind mit Malereien und Wandteppichen
  geschmückt. Der Boden ist mit dicken Teppichen, Fellen und zu den
  Wänden hin mit einigen Kissen ausgelegt.

  Bei diesem Raum könnte es sich um das Lager von Menschen, Halbmenschen
  oder Humanoiden handeln. Dann wäre hier auch ein entsprechender Schatz
  zu finden. Ansonsten finden sich hier nur ein paar Nahrungsvorräte,
  zwei Fässer mit Trinkwasser, einige Silbermünzen und je nach Ebene
  vielleicht auch ein paar Goldmünzen. Bei einem Wurf von 1-3 kommt es
  hier nach wenigen Minuten zu einer Zufallsbegegnung.
\\
7 & 
  \ldots{} findet Ihr einen kreisrunden Raum von etwa 6 Metern
  Durchmesser. In der Mitte des Raumes befindet sich ein kreisrundes
  Becken, welches mit einer dunklen wässrigen Flüssigkeit gefüllt ist.
\\
8 & 
  \ldots{} entdeckt Ihr eine Wendeltreppe, die sowohl nach unten als
  auch nach oben führt. Nach etwa 6 Metern rauf bzw. runter erreicht ihr
  eine verschlossene Tür. Weiter mit 1W10 auf dieser Tabelle.
\\
9 & 
  \ldots{} kommt ihr in einen Gang, in dem die Luft deutlich kühler ist,
  als vor der Tür nach etwa 6 Metern durchschreitet Ihr einen kleinen
  Torbogen und erreicht eine Plattform von etwa 6 x 9 Metern Größe. Die
  durch eine niedrige steinerne Brüstung begrenzt ist. Die Plattform
  ragt in eine gewaltige natürliche Höhle hinein. Als Ihr Euch herum
  dreht, seht ihr, dass die Wand mit dem Torbogen, weit nach oben
  reicht. Mit euren Lichtquellen ist das obere Ende der Wand, bzw. das
  Dach der Höhle nicht zu erkennen. Unterhalb der Plattform scheint es
  in unendliche schwarze Tiefe hinab zu gehen. Links und rechts von der
  Plattform ist ebenfalls nichts zu sehen als die schroffe senkrechte
  Felswand.

  Genau in der Mitte der Plattform, zum Abgrund hin, ist die Brüstung
  auf einer Länge von etwa 3 Metern unterbrochen. Links und rechts
  bewachen steinerne Gargylen die Öffnung in den Abgrund.

  Ihr spürt etwas Luftbewegung um Euch herum.

  Hier ist vieles möglich. Es könnte sich einfach um das handeln, was es
  ist. Eine seltsame Plattform mit Steinskulpturen. Oder eine Falle: die
  Gargylen warten bis ein unvorsichtiger Charakter an die Kante tritt,
  und stürzen ihn dann in die Tiefe, um sich später über ihn her
  zumachen. Oder es könnte eine Art Haltestelle für ein unterirdisches
  Verkehrsmittel sein, welches mit Hilfe von Flugsauriern oder Drachen
  betrieben wird. Der Ritt auf der Drachenbahn könnte überall hin
  führen. Auf ferne Inseln, ferne Planten, seltsame Ebenen \ldots{}
\\
\end{tabularx}
\begin{tabularx}{\columnwidth}{cZ}
10 & 
  \ldots{} kommt ihr in einen Raum, bei dem es sich unverkennbar um eine
  Gruft handelt. Durch einen kleinen quadratischen Vorraum tretet Ihr
  durch zwei Säulen hindurch, und erreicht eine breite Treppe, die etwa
  einen Meter abwärts führt. Hier findet ihr einen rechteckigen Raum von
  etwa 9 x 18 Metern Größe, in dem mehrere steinerne Särge stehen.

  Ganz offensichtlich handelt es sich hier um das Lager von Untoten.
\end{tabularx}

\subsubsection{V - Nachdem Ihr einige Minuten mühsam dem Verlauf der Höhle
gefolgt seid \ldots{} (1W12)}

\begin{tabularx}{\columnwidth}{cZ}
1 & \ldots{} erreicht ihr einen kühlen unteridischen Fluss. Ein Stück weit
  folgt er dem Verlauf der Höhle, dann verschwindet er durch einen
  Tunnel nach links/rechts. Es sieht fast so aus, als könnte man durch
  den Tunnel hindurch schwimmen.

  Wenn Ihr den Fluss durchwatet, könnt Ihr auf der anderen Seite über
  ein paar ansteigende Felsvorsprünge weiter klettern.  \\

2 & \ldots{} erreicht Ihr eine mächtige Halle. Hier und da funkeln
  Kristalle im Gestein weit über Euch. Ihr seht Gruppen von imposanten
  Tropfsteinsäulen, und riesige Kaskaden von feucht glänzendem Stein, so
  als sei hier vor Jahrtausenden ein Wasserfall versteinert worden.

  Aber was war das? War da nicht ein Knurren zu hören, hinter einer der
  großen Tropfsteinformationen?

  Eine Erkundung der Halle ergibt 1-4 Tunnel, die in benachbarte Höhlen
  führen.  \\
3 & \ldots{} findet ihr in einer Nische unerwartet behauenen Stein, und
  \ldots{} eine Tür - seltsam.

  Würfel 1W10, es geht weiter auf Liste IV.  \\

4 & \ldots{} kommt Ihr an eine große Wasserfläche in einer gewaltigen
  Höhle. Als Ihr dicht an das Wasser tretet, seht ihr im Fackelschein
  einige Grottenolme, die sich schnell aus Eurem Blickfeld flüchten. In
  einiger Entfernung seht ihr aus dem Wasser einen Felsen aufragen -
  eine kleine Insel.

  Auf der Insel könnte sich verschiedenes finden:

  \begin{enumerate}
    \item Nichts, nur blanker Fels,
    \item das Lager eines seltsamen Höhlenbewohners,
    \item ein Schatz, der vielleicht durch einen Zauber getarnt ist,
    \item einen geheimen Ort für besondere Rituale, vermutlich sind noch
    \item Spuren der letzten Beschwörung zu entdecken, oder schließlich:
    \item das \textbf{Ziel der Expedition}.
  \end{enumerate}

  Die Höhle ist wirklich riesig, so dass hier viel zu suchen und
  vielleicht auch einiges zu entdecken ist. Ansonsten hat sie 0-3
  (1W4-1) natürliche Ausgänge.
\\
5 & 
  \ldots{} erreicht ihr einen Höhle, die aus grob gebrochenem Granit
  besteht. Überall seht ihr Spalten und Fissuren, zum Teil so groß, dass
  man sich hindurch quetschen könnte, zum Teil zu eng um weiter zu
  kommen. Wenn Ihr Euren Weg hier fortsetzen wollt, müsst ihr über eine
  etwa 20 Meter breite stark abschüssige Granitfläche überqueren. Am
  unteren Ende der Fläche seht ihr auch mit Euren Lichtquellen nur
  gähnende Dunkelheit.

  Rettungswurf Lähmung um nicht abzustürzen. Der Fall in die Dunkelheit
  ist 3 bis 18 Meter tief und macht entsprechend 1W6 bis 6W6 Punkte
  Schaden.

  Gelingt es in der Tiefe Licht anzuzünden - und wurde der Sturz
  überlebt - könnte hier unten \textbf{ein besonderer Schatz} gefunden
  werden. Oder sogar \textbf{das Ziel der Expedition}. Ansonsten geht es
  weiter mit 1W10 auf dieser Liste.
\\
6 & 
  \ldots{} hört Ihr ein Geräusch. Ein Wurf auf der passenden
  Zufallsbegegnungstabelle zeigt, worum es sich handelt.
\\
7 & 
  \ldots{} findet Ihr Spuren. Vielleicht ein paar Fußabdrücke im Staub
  des Höhlenbodens, etwas Kot, oder auch den Kadaver eines Opfers. Ein
  Wurf auf der Zufallsbegegnungstabelle zeigt, was die Spur hinterlassen
  hat.
\\
8 & 
  \ldots{} bemerkt Ihr eine Bewegung vor Euch. Eine Zufallsbegegnung.
\\
9 & 
  \ldots{} werdet Ihr plötzlich von einem Angriff überrascht.
\\
10 & 
  \ldots{} zweigt sich der Tunnel durch den Ihr gerade geht
  ypsilon-förmig auf. Folgt Ihr dem linken, oder dem rechten Gang?
\\
11 & 
  \ldots{} endet der Weg vor Euch. Ihr seid in eine Sackgasse geraten.

  Bei 1 auf 1W6 erwartet die Abenteurer nach wenigen Minunten des
  RÜckwegs eine Zufallsbegegnung. \\

\end{tabularx}

\section{Ein Kampagnen-Generator}
\by{Wanderer Bill}

Dieser Generator dient dazu Hexfeld-Karten für Sandbox-Abenteuer zu
erzeugen. Die meisten Generatoren versuchen nach einem bestimmten
Algorithmus ein Hexfeld nach dem anderen zu generieren. Ich habe
damit auch viel rum probiert, aber ich finde es kommen dabei oft
sehr beliebige, inkonsistente Karten heraus, und es dauert auch viel
zu lange. Daher versuche ich mit den folgenden Tabellen eher eine
Landschaft in groben Strichen zu erzeugen. Mit ein - zwei Würfen
wird ein ganzer Landstrich platziert und mit einem inspirierenden
Namen versehen. Mit weiteren Würfen, wird die Ausdehnung eines
Landstrichs, sowie Distanz und Richtung zu anderen Landstrichen
festgestellt. Schließlich werden das Ausgangsdorf, besondere Orte,
und der Weg dorthin ermittelt. Dabei kann es durchaus sein, dass die
Wegdauer und die geografische Lage zweier Orte nicht so recht
zusammen passen wollen. Aber solche Diskrepanzen sind durchaus
gewollt, und sollten als Hinweise auf ungewöhnliche Hindernisse,
oder vielleicht magische Beförderungsmittel gelesen werden.

\subsection{36 fantastische Landstriche}
\begin{tabularx}{\columnwidth}{cZcZ}
\textbf{W66} & \textbf{Landschaft} & \textbf{W66} &
\textbf{Landschaft} \\ 
11 & Nebelmoor & 41 & das Brachfeld\\
12 & Grauwald & 42 & das Dunkeltal\\
13 & Düsterwald & 43 & das Bruchtal\\
14 & Feuerküste & 44 & die hohen Recken\\
15 & Eisküste & 45 & die milden Hügel\\
16 & die süßen Felder & 46 & das endlose Eisfeld\\
21 & Eiszacken & 51 & Südmeer\\
22 & Nebelberge & 52 & Nordmeer\\
23 & die schwarzen Berge & 53 & Ostmeer\\
24 & die eisernen Berge & 54 & Westmeer\\
25 & das Wetterfenn & 55 & Dunkelsee\\
26 & das Schwarzmoor & 56 & Purpursee\\
31 & die gelbe Steppe & 61 & Silbersee\\
32 & die weite Wüste & 62 & Ozean des Morgens\\
33 & die Schwebeberge & 63 & Ozean der Stille\\
34 & die Salzweite & 64 & die Wetterinseln\\
35 & der Bruch & 65 & die Inseln der Hoffnung\\
36 & die Wirrweite & 66 & der Abgrund\\
\end{tabularx}


\subsection{Ausdehnung und Fläche von Landstrichen}
\begin{tabularx}{\columnwidth}{cccZ}
\textbf{1W3} & \textbf{1W6} & \textbf{Hexfelder} &
\textbf{Ausdehnung} \\
1 & 1 & 2 x 2 & 144 Quadratmeilen oder 400 qkm\\
1 & 2 & 2 x 4 & 288 Quadratmeilen oder 800 qkm\\
1 & 3 & 2 x 6 & 432 Quadratmeilen oder 1.200 qkm\\
1 & 4 & 2 x 8 & 576 Quadratmeilen oder 1.600 qkm\\
1 & 5 & 4 x 4 & 576 Quadratmeilen oder 1.600 qkm\\
1 & 6 & 4 x 6 & 864 Quadratmeilen oder 2.400 qkm\\
2 & 1 & 4 x 8 & 1.152 Quadratmeilen oder 3.200 qkm\\
2 & 2 & 6 x 6 & 1.296 Quadratmeilen oder 3.600 qkm\\
2 & 3 & 4 x 10 & 1.440 Quadratmeilen oder 4.000 qkm\\
2 & 4 & 6 x 8 & 1.728 Quadratmeilen oder 4.800 qkm\\
2 & 5 & 6 x 10 & 2.160 Quadratmeilen oder 6.000 qkm\\
2 & 6 & 8 x 8 & 2.304 Quadratmeilen oder 6.400 qkm\\
3 & 1 & 6 x 12 & 2.592 Quadratmeilen oder 7.200 qkm\\
3 & 2 & 8 x 10 & 2.880 Quadratmeilen oder 8.000 qkm\\
3 & 3 & 8 x 12 & 3.356 Quadratmeilen oder 9.600 qkm\\
3 & 4 & 10 x 10 & 3.600 Quadratmeilen oder 10.000 qkm\\
3 & 5 & 8 x 14 & 4.032 Quadratmeilen oder 11.200 qkm\\
3 & 6 & 8 x 16 & 4.608 Quadratmeilen oder 12.800 qkm\\
\end{tabularx}

Für kleine Hexkarten sollte nur nur mit einem Würfel 
auf den 6 obersten Einträgen gewürfelt werden.

\subsection{Himmelsrichtungen und
Distanzen}\label{himmelsrichtungen-und-distanzen}

\subsubsection{Himmelsrichtungen bei horizontalen Hexfeldern}
\begin{tabularx}{\columnwidth}{cZcZ}
1 & Nordwesten & 4 & Südosten\\
2 & Westen & 5 & Osten\\
3 & Südwesten & 6 & Südosten\\
\end{tabularx}

\subsubsection{Himmelsrichtungen bei horizontalen Hexfeldern}
\begin{tabularx}{\columnwidth}{cZcZ}
1 & Norden & 4 & Süden\\
2 & Nordwesten & 5 & Südosten\\
3 & Südwesten & 6 & Nordosten\\
\end{tabularx}

\subsection{Das Ausgangsdorf befindet
sich}

\begin{tabularx}{\columnwidth}{cZ}
1 &  an der Küste \\
2 &  auf einer Insel die Teil eines Archipels ist \\
3 &  in einem engen Tal im Gebirge \\
4 &  auf einer Insel in einem großen Binnensee \\
5 &  an einer Furt durch einen Fluss \\
6 &  am Rand der Wüste \\
7 &  in einer Oase \\
8 &  am Rande des ewige Eises\\
\end{tabularx}

\subsection{36 fantastische Orte}

\begin{tabularx}{\columnwidth}{cZ}
\textbf{W66} & \textbf{Ort} \\
11 & Die Höhlen von/des/der/am/im \ldots{} (Ort oder
Landstrich)\\
12 & Die schwebende Stadt \emph{Aereon}\\
13 & Die ewige Stadt \emph{Panrogai}\\
14 & Die Minen von/des/der/am/im \ldots{} (Ort oder
Landstrich)\\
15 & Die Eisenfelder\\
16 & Der Turm des/der \ldots{} (NSC, Monster)\\
21 & Die hängenden Gärten von \ldots{} (NSC, Ort oder
Landstrich)\\
22 & Die Pyramiden zu/des/der \ldots{} (NSC oder Ort)\\
23 & Die Nebelfelder\\
24 & Der Wald ohne Wiederkehr\\
25 & heiße Quellen\\
26 & der Übergang\\
31 & der Thing-Stein\\
32 & die Katakomben von/des/der/am/im \ldots{}\\
33 & \emph{Hara}, die Stadt der Händler\\
34 & die Walddörfer\\
35 & die lebenden Bäume\\
36 & die lebenden Berge\\
41 & der sprechende Fluss\\
42 & der Monolith\\
43 & Die Quelle der Erneuerung\\
44 & das große Tal\\
45 & der Gipfel der Weitsicht\\
46 & das grüne Tal\\
51 & der Weltenbaum\\
52 & die alte Esche\\
53 & die Eiche der Vorfahren\\
54 & der Friedensstein\\
55 & die Schwefelbäume\\
56 & die Hallen von/des/der/am/im \ldots{}\\
61 & das Tal der Alten\\
62 & der gefallene Stern\\
63 & das Höllentor\\
64 & das Feentor\\
65 & der Ursprung\\
66 & das letzte Ende\\
\end{tabularx}

\subsection{Von A nach B ...}

\begin{tabularx}{\columnwidth}{cZ}
%\caption{Distanzen und Reisedauer}\\
2 & eine Stunde, gleiches Hexfeld\\
3 & eine Stunde, benachbartes Hexfeld\\
4 & wenige Stunden, benachbartes Hexfeld\\
5 & eine halbe Tagesreise, 1-2 Hexfelder\\
6 & eine Tagesreise, 2-3 Hexfelder\\
7 & zwei Tagesreisen, 4-6 Hexfelder\\
8 & drei Tagesreisen, 6-9 Hexfelder\\
9 & eine Woche, 10-18 Hexfelder\\
10 & 2-3 Wochen, 19-54 Hexfelder\\
11 & einen Monat, 40-72 Hexfelder\\
12 & zwei Monate, 60-150 Hexfelder\\
\end{tabularx}

Der jeweils untere Bereich der Reisedauer legt sehr beschwerliche
Bedingungen nahe, vielleicht dichten Urwald, schroffe Klippen, tiefe
Täler, eisige Hochebenen. Der obere Bereich der Reisedauer legt günstige
Bedingungen nahe, eine gut ausgebaute Straße vielleicht, oder ein Fluss
auf dem die Reise zügig voran geht.

\subsection{Lokale Konflikte}

Lokale Konflikte können Aufhänger für Szenarien und Abenteuer
bieten. Würfele für jeden Ort ein bis zwei mal, und kombiniere die
Ergebnisse.

\begin{tabularx}{\columnwidth}{cZ}
1 & Die Ernten werden jedes Jahr magerer.\\
2 & Die Söldner des lokalen Herrschers treiben unerbittlich hohe
Steuern ein. \\
3 & [Humanoide] aus der nahe gelegenen Wildnis wagen immer kühnere
Überfälle.\\
4 & Es kommt immer wieder zu leichten Erdbeben, Unheil liegt in der
Luft.\\
5 & Irgendwo in der Nähe muss ein [Monster] ein neues Lager bezogen
haben. Jäger und Wanderer sind spurlos verschwunden.\\
6 & Es gibt Streit über den Dorf-Vorstand.\\
7 & Der alte Druide ist schwer krank. Wer kann ihn retten?\\
8 & Ein Bauer bringt in letzter Zeit immer wieder überdimensional
großes Gemüse mit zum Markt. Geht das mit rechten Dingen zu?\\
9 & Jemand, oder etwas, hat die Brunnen vergiftet. Viele sind krank,
einige gestorben.\\
10 & Brandstifter gehen um.\\
11 & Ein Gerücht geht um: die Kräuterfrau [oder ein beliebiger
anderer NSC] hat sich in ein [Monster] verwandelt.\\
12 & Die [Halbmenschen] aus der Nachbarschaft sind verärgert, und
verlangen Kompensation (würfele noch einmal um mehr über die Ursache
des Konflikts zu erfahren).\\
\end{tabularx}

\subsection{Entfernte Konflikte}

Entfernte Konflikte können Hinweise auf das Thema, oder den
übergeordneten Plot der Kampagne geben. Die Klärung eines
entfernten, oder überregionalen Konflikts, könnte das 
große Ziel oder das Ende einer Kampagne signalisieren.

Mehrmals würfeln und kombinieren!

\begin{tabularx}{\columnwidth}{cZ}
1 & Truppen werden für einen Feldzug zusammen gezogen. Es wird Krieg
geben.\\
2 & Ein mächtiges [Monster] sucht das Land heim. Man sagt, es käme
nicht von dieser Welt.\\
3 & Eine Hungersnot ist ausgebrochen.\\
4 & Hunderte [Menschen/Halbmenschen/Humanoide] fliehen nach
[Richtung / Landstrich].\\
5 & Das Alte Reich ist zusammen gebrochen.\\
6 & Die Mächte des Chaos haben das Land übernommen.\\
7 & Ganze Dörfer sind zu Staub zerfallen.\\
8 & Der König ist tot!\\
9 & Der Mond kommt jeden Tag näher.\\
10 & Ein Monolith aus einer anderen Dimension bringt Verderben über
die Welt.\\
11 & Wo die Stadt war ist nur noch ... Nichts.\\
12 & Die Sümpfe breiten sich aus.\\
13 & Es hat seit Monaten nicht mehr geregnet.\\
14 & Millionen von [Insekten] suchen das Land heim.\\
15 & Die Weisen der drei Türme sind in erbitterten Streit geraten.\\
16 & Ein uraltes Protal ist wieder geöffnet.\\
17 & Ein Himmelskörper ist eingeschlagen. Die Staubwolke ist über [Distanz] Meilen zu
sehen. Der Krater hat uralte Dinge frei gelegt.\\
18 & Seit Wochen liegt das Land unter unnatürlichen
Gewitterwolken.\\
19 & Ein Himmelsschiff ist gelandet.\\
20 & Eine Epidemie verwüstet das Land.\\
\end{tabularx}

\subsection{Fraktionen}

\begin{tabularx}{\columnwidth}{cZ}
1 & Die Bernstein-Bruderschaft\\
2 & Die Bruderschaft von Dolch und Krone\\
3 & Die weiße Legion\\
4 & Die Füchse von Chele\\
5 & Elells Gilde\\
6 & Die Stäbe der Wene\\
7 & Die Gesellschaft der Smaragdmünze\\
8 & Joanes Wölfe\\
9 & Die Kompanie des Zepters\\
10 & Das Bündnis von Pfeil und Zepter\\
11 & Die Pfeile von Wastow\\
12 & Die Wanderer von Tumunzar\\
13 & Hilies Bündnis\\
14 & Die Heiligen von Ismud\\
15 & Die Drachen des Karmesin\\
16 & Die Speere von Badun\\
17 & Die Smaragd-Gesellschaft\\
18 & Die Azurschurken\\
19 & Die Silberne Gemeinschaft\\
20 & Der Orden der Saphirstäbe\\
21 & Die Legion des Purpurwolfs\\
22 & Der Orden des Azurdolchs\\
23 & Die Heiligen von Tirione\\
24 & Die Azurblauen Schilde\\
25 & Die Gesellschaft des Schattenhammers\\
26 & Die Gemeinschaft des Eisernen Pfeils\\
27 & Die Gemeinschaft des Silbernen Raben\\
28 & Der Purpurrat\\
29 & Die Kompanie des Schwertes und der Münze\\
30 & Der Kreis von Zahn und Zepter\\
\end{tabularx}

\section{46.656 Psychedelische Landschaften}
\by{Wanderer Bill}

Die folgenden Tabellen könntest Du für Deine nächste Science
fiction-Kampagne, oder die nächste Reise durch seltsame Ebenen
verwenden.

Ich habe diese Tabellen während meinen Vorbereitungen für eine
Traveller5-Kampagne angelegt, aber das nur nebenbei. Diese
psychedelischen, futuristischen Fantasielandschaften aus den 1960er
bis 1980er Jahren faszinieren mich. Es war eine Zeit, als Fantasy 
und Sci-Fi noch keine getrennten Genres waren, alles war \textit{Science fantasy}.
Such doch mal im Netz nach ``science fiction landscape'' oder
``psychedelic landscape'' ... cool, oder?

\begin{tabularx}{\columnwidth}{cZZZ}
\textbf{W66} & \textbf{Das Land ...} & \textbf{Der Himmel ...} &
\textbf{In der Ferne ...} \\
11 & geronnenes Karamell & grau-violett & eine kugelförmige Raumstation \\
12 & blutrote gezackte Hügel & wie dickes dunkles Blut & ein
majestetischer Drache \\
13 & bunte würfelförmige Hügel, wie Süßigkeiten & ein tieferes Blau
& das schemenhafte Bild eines Ringplaneten \\
14 & weite grüne Hügel & grauer Kohlenstaub & eine röhrenförmige
Struktur \\
15 & ein saftiger tropischer Regenwald & smaragd-grün & Wolken
winziger animierter Wesen, Insekten? Naniten? \\
16 & endlose Muster industrieller Strukturen & ein Verlauf von
Kobaltblau zu glitzerndem Türkis & ein Mond der zu nahe erscheint.
\\
21 & endlose Dünen aus Industriemüll & schwefelgelb & einige
Raumfähren \\
22 & flach mit gelegentlichen geometrischen Figuren & ein monotones
Hellblau & eine diskusförmige Raumstation \\
23 & beunruhigend surreal & sepiafarben mit pinken Zirhuswolken &
fledermausartige Flugtiere \\
24 & eine rauhe Wüste & wie ineinander verlaufende Tusche &
unheilvolle dunkle Wolken \\
25 & sumpfig mit gelegentlichen Hügeln & fast weiß & die
würfelförmige Silhuette des Orbitalhafens \\
26 & ein türkisblaues Meer mit steilen hügeligen Waldinseln &
leuchtendes Blau mit schwebenden Eiskristallen & eiförmige Kapseln,
die lautlos unsichtbaren Bahnen durch den Himmel folgen \\
\end{tabularx}

\begin{tabularx}{\columnwidth}{cZZZ}
\textbf{W66} & \textbf{Das Land ...} & \textbf{Der Himmel ...} &
\textbf{In der Ferne ...} \\
31 & ein endloser sturmgepeitschter Ozean & ein kalter blauer Nebel
& ein imposantes Luftschiff \\
32 & eine wuchernde Metropole & ein trübes Rot & große vogelartige
Kreaturen, die im Sonnenuntergang ihre Nester aufsuchen. \\
33 & wellige Hügel aus reflektierendem Metall &
schwarze Ewigkeit, ein endloses Sternenfeld & giftige
industrielle Rauchschwaden. \\
34 & ein seltsam friedliches Idyll & mit dichten Wolken zugezogen &
eine riesige kugelförmige Struktur \\
35 & ein Labyrinth aus tiefen Spalten und Schluchten & violett mit
gelben Wolken & ein perfekter Regenbogen. \\
36 & Hügel wie aus gebranntem Ton und ein meandernder Fluss aus
Quecksilber & ein seltsamer vielfarbiger Dunst & ein Streifen grünen
Lichts. \\
41 & ein Netzwerk aus vielschichtigen länglichen Strukturen  & ein
Perfekter Gradient von Blautönen & zwei untergehende Sonnen. \\
42 & ein verwesender urzeitlicher Wald & vollgesogen mit
Feuchtigkeit & die abnehmende Sichel eines nahen Mondes. \\
43 & eine weite gelbe Steppe, die nach Teer riecht & ein Aufruhr aus
rötlichen Gaswolken & die Silhuetten einiger schwebender Inseln. \\
44 & schillernde Weite aus transparentem Laub & voller schwebender
Pflanzensamen & der helle Schein des galaktischen Kerns. \\
45 & eine endlose Ebene aus feinem weißem Sand & violett und schwarz
& ein Regenschauer \\
46 & vereinzelte große Weidetiere auf violetten Grashügeln & ein
Blätterdach aus Sternen & farbenfrohe Reflektionen. \\
\end{tabularx}

\begin{tabularx}{\columnwidth}{cZZZ}
\textbf{W66} & \textbf{Das Land ...} & \textbf{Der Himmel ...} &
\textbf{In der Ferne ...} \\
51 & rostfarbene schroffe Berge & ein Gradient von nebeligem Lila zu
dunklem Blau
& federartige schwebende Partikel, die ein unwirkliches Licht reflektieren \\
52 & aus dem Nebel hervorragende turmartige Säulen & vereinzelte
graue Wolken und Sonnenstrahlen & eine dunstige Vorahnung von dem
was morgen kommt. \\
53 & ein Wüste aus staubigen, schroffen Kratern & eine perfekter
Gradient von Dunkelblau zu fast perfektem Weiß & eine Nebelbank. \\
54 & Dünen aus farbigem Sand, wie gemahlener Marmor & ein stumpfes
Grau &
bunte, in Sonnenlicht getauchte Wolken. \\
55 & ein Meer weißer Dünen & ein Gradient von Hellblau zu tiefem
Schwarz &
der Rauchschweif eines Raumschiffes, welches in den Himmel jagt. \\
56 & organisch anmutende, gewölbte Formationen & ein verwirrendes
helles Gelb &
massive, quaderförmige schwebende Habitate. \\
61 & halb-flüssige farbige Flächen & Wirbel aus farbigen Gasen &
chromatische Reflektionen eines tief fliegenden Raumschiffes. \\
62 & weite Terrassen aus grauem Schiefer & ein niedriger blauer
Nebel & die Ahnung des tiefen Weltraums. \\
63 & driftende Felsinseln in einem Meer aus Lava & ein stumpfes,
monotones Hellblau & ein paar pinke, reptilienartige Flugtiere \\
64 & ein FLussdelta mit Mangroven und vereinzelten Dörfern & ein
Verlauf von Orange zu Rot & die blasse Silhuette einer nahen
künstlichen Welt. \\
65 & eine visköse, ölige Oberfläche & ein Gradient von Schwefelgelb
zu Kobaldblau & die stalagmiten-artigen Türme der Arkologie \\
66 & monumentale schneebedeckte Berge & wie mit roten Streifen
bemalt &
der leuchtende Wirbel der Galaxie. \\
\end{tabularx}

